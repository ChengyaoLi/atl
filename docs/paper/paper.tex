% DOCUMENT CLASS
\documentclass[letterpaper, 10 pt, conference]{ieeeconf}
\IEEEoverridecommandlockouts
\overrideIEEEmargins  % required to meet printer requirements.

% DOCUMENT STRUCTURE
\usepackage{subfiles}

% TABLES
\usepackage{caption}
\usepackage{multirow}

% ALGORITHMS
\usepackage{algorithmic}
\usepackage{algorithm}

% GRAPHICS
\usepackage{float}
\usepackage{subcaption}
\usepackage{tikz}
\usepackage{tikzscale}
%\usepackage{tkzgraph}
\usetikzlibrary{matrix}
\usepackage{verbatim}
\usetikzlibrary{arrows,shapes}
\usetikzlibrary{calc}
\usepackage{graphicx}
\usepackage{hyperref}

% MATH SYMBOLS
\usepackage{amsfonts} % mathbb{R}
\usepackage{amsmath}
\DeclareMathOperator*{\argmin}{arg\,min}

% COMMENT BLOCKS
\usepackage{verbatim}

% SUBSCRIPT COMMANDS
\newcommand{\superscript}[1]{\ensuremath{^{\textrm{#1}}}}
\newcommand{\subscript}[1]{\ensuremath{_\textrm{#1}}}

% TITLE
\title{\large \textbf Autonomous Landing of a Quadrotor on a Moving Platform in
Outdoor Environments}

% AUTHORS
\author{\
Chris Choi\authorrefmark{1}, Stanley Brown\authorrefmark{2} and  Steven L. Waslander\authorrefmark{3}
\thanks{\superscript{*} M.A.Sc. Candidate, Mechanical and Mechatronics Engineering, University of Waterloo; c33choi@uwaterloo.ca}
\thanks{\superscript{2} M.A.Sc. Candidate, Mechanical and Mechatronics Engineering, University of Waterloo; s52brown@uwaterloo.ca}
\thanks{\superscript{\dag} Assistant Professor, Mechanical and Mechatronics Engineering, University of Waterloo; stevenw@uwaterloo.ca}
\vspace{0.5in}
}

\begin{document}
\maketitle
\thispagestyle{empty}
\pagestyle{empty}

% ABSTRACT
\begin{abstract}
The task of landing an aircraft on moving objects has been a topic of interest
ever since humans took to the skies in the early twentieth century. Typically
when attempting landing small, Micro UAVs such as quadrotors, the landing
procedure is either done manually, or using GPS guidance to land on a
stationary target. If landing on a moving target or in high wind conditions, a
human operator is generally required. However this requires the human operator
to have a clear line of site to both the target and the aircraft during all
stages of the landing procedure. In recent years there have been several papers
that propose a set of methods for landing on a slow moving platform in ideal
conditions such as low wind and ideal GPS signals with little or no noise. Our
contribution in this work is the demonstration of a set of algorithms that
allow a quadrotor to perform a autonomous landing on a car travelling at over 20
meters per second while relying only on visual inertial data collected from and
on-board gimbaled camera system. From this image data, along with on-board IMU
readings, it is shown that both speed of the target and the quadrotor in the
inertial frame can be directly estimated, and using this information a set of
controllers and planners can be developed that allows for the quadrotor to
perform a high speed landing manoeuvre in a variety of conditions. The
robustness of our system is demonstrated by landing in without using GPS,
poorly lit conditions, and with high winds.
\end{abstract}


% INTRODUCTION
\section{Introduction}
Over the years there has been a growing interest in autonomous landing of
quadrotors, as landing is often one of the most dangerous and risk prone parts
of flight. There are now several commercial and research based implementations
of autonomous landing with both the DJI Phantom and Pixhawk flight controllers
being capable of automatic landing using a series control algorithms. However,
there does not appear to be a reliable method of landing a quadrotor on either
a moving platform or in relatively large wind disturbances.

In recent years there have been at least least 5 academic publications on the
topic~\cite{Lee2012, Kim2014, Voos2010, Friis2009, Ling2014, Herisse2012} that
approach the problem from a variety of angles. Overall none of these approach
have completely solved the problem and most approaches only work on slow moving
platforms with little or no wind disturbances.

One of the most challenging problems associated with autonomous landing is the
difficulty of obtaining a reliable state estimate of the landing target
relative to the quadrotor. This is due to the fact that the landing target can
often go out of the camera's field of view causing the quadrotor to lose track
of the target or measurements can become corrupted by vision based artifacts
that include lighting, lens distortion, color, etc. In many cases GPS is not a
viable option both the landing target and the quadrotor due a lack of accuracy
unless an RTK GPS system is used. In many cases the use of an RTK system is not
a readily available option due to cost and the lack of required equipment.

% RELATED WORK
\section{Related Work}
The autonomous landing problem can be thought of as a set of three separate
control problems or stages. First the quadrotor must detect the relative
position and velocity of the landing target using either GPS of visual methods.
Next the quadrotor must determine a rendezvous location with the landing target
and plan the required flight trajectory. Once the quadrotor has rendezvoused a
final landing trajectory must be calculated between quadrotor and the landing
pad.

For the perception portion of the problem there have been two main techniques
use to identify the landing pad and estimate its state relative to the
quadrotor. In the work of Kim, 2014~\cite{Kim2014} a basic colour thresholding
technique was used to identify the landing target and the MIT AprilTags library
\cite{apriltags} was used to provide state estimations. In the work of Heriesse
Et al., 2012~\cite{Herisse2012} an optical flow technique was applied to a
series of images captured on-board the quadrotor to estimate the required
states required to control the quadrotor and conduct autonomous landing. While
both of these techniques have been shown to work in very specific cases both,
require case specific parameters to be applied ahead of time, thereby limiting
there general applicability.

In the case of Ling, 2014~\cite{Kim2014}, the brightness used in his
thresholding technique must be set ahead of time and is sensitive to lighting
changes. A similar issue is noted in the work of Heriesse Et al., 2012
\cite{Herisse2012}. Moreover in the work of Ling, there were implementation
issues due to the slow state estimation rate of the AprilTag library and the
AprilTag falling out of view of the camera during the final stages of decent.
This meant that as the quadrotor approached the landing pad it tended to lose
any measurements of its state and therefore had complete the final landing
procedure by propagating the last measured state forward in time.

Examining the related literature on the controllers used in solving the the
autonomous landing problem, the majority of literature either uses a set of PID
or a Linear Quadratic Controllers (LQC) to command and control the quadrotor at
all stages of flight. A LQC was thoroughly explored in Friis Et al., 2009
\cite{Friis2009} but the authors admit even though LQC performed slightly
better than the PID controller the difference was minor and may be attributed
to the additional time spent tuning the LQC\@. In~\cite{Herisse2012}, a PID
controller was developed to land on a vertically oscillating platform no
lateral movement. While interesting, the solution took over 1 minute to
transition from hovering over the landing pad to landing. Additionally the
experiments did not seem to account for pitch or roll of the platform.

In this project we focus developing a set of controllers that produce the final
landing trajectory and using a  AprilTags~\cite{apriltags} to provide the
estimated state of the quadrotor relative to the landing pad. We seek to
develop both a more robust and rapid method of determining the state of the
quadrotor and landing pad system and to develop a set of PID controllers that
can be used to control the quadrotor based on the measured state information.


% CONTRIBUTION
\section{Contribution}

\begin{itemize}
  \item{Robust Landing}
  \item{AprilTag estimation with Kalman Filter}
  \item{AprilTag windowing to speed up detection}
  \item{Light Invariant AprilTag detection}
  \item{Tracking Controller}
  \item{Trajectory Planning}
\end{itemize}

% PROBLEM FORMULATION
\section{Problem Formulation}

% CAMERA TRANSFORMS
\section{Camera Transforms}

% GIMBAL TRANSFORMS
\section{Gimbal Transforms}


% LANDING TARGET ESTIMATION
\section{Landing Target Estimation}

Kalman Filter constant acceleration

\subsection{Adaptive Image Processing}
As discussed previously, the low rate of the AprilTag library when processing
images of 640 by 480 pixels results in an update rate of approximately 3 to 5
fps on the on-board computers used on the target quadrotor, which is too low to
be used for controlling the quadrotor. Building upon who improved the AprilTag
estimation rate by reducing the brightness of the image  such that the majority
of the image is black, we implemented an adaptive down sampling of the image size
as the estimated distance between camera and AprilTag decreased. In this work
two image down sampling sizes $320 \times 280$ (half resolution) and $160
\times 140$ (quarter resolution), were used. Whenever the distance between the
camera and the AprilTag is less than 1.5 and 3 meters, the images are down
sampled to the two resolutions respectively. If the camera is further than 3
meters from the AprilTag, the native resolution of $640 \times 480$ is used.
This change only affected the image processing time when the quadrotor is
within 3 meters of the quad and has the desired effect of maintaining a high
AprilTag estimation rate.

\subsection{Adaptive Image Windowing}
If the AprilTag in the image space does not translate too suddenly, one can
make the assumption that the previous detected AprilTag will approximately be
in the same approximate image coordinates. A bounding box around the AprilTag
can be formed and used to feed into the AprilTag detector.

\subsection{AprilTag Inception}
To address both the decreasing state update rate as a function of proximity and
reduce the probability of losing sight of the AprilTag during landing a
secondary AprilTag is embedded in the larger AprilTag, we term this technique
AprilTag Inception (patent pending). This secondary AprilTag is assigned
different family ID to avoid confusion during detection, and is placed at the
center of the larger primary AprilTag. Whenever the secondary AprilTag is
captured in the image, the adaptive windowing method is set to track only the
secondary AprilTag rather than the larger one, which reduces the portion of the
image that must be processed in with each image capture. An example of this
procedure is highlighted in Figure.

\section{Light Invariant AprilTag}


% QUADROTOR CONTROL
\section{Quadrotor Control}


% TRAJECTORY PLANNING
\section{Trajectory Planning}

\begin{equation}
  \textbf{x} = \begin{bmatrix}
      x_{1} & x_{2} & x_{3} & x_{4} & x_{5}
    \end{bmatrix} \\ 
    = \begin{bmatrix}
      x & \dot{x} & z & \dot{z} & \theta
    \end{bmatrix} \in {\rm I\!R}^{5}
\end{equation}

\begin{equation}
  \textbf{x}_{d} = \begin{bmatrix}
      x_{d} & \dot{x_{d}} & z_{d} & \dot{z_{d}} & \theta_{d}
  \end{bmatrix}
\end{equation}

\begin{equation}
  \textbf{u}(t)
    = \begin{bmatrix} u_{1} \\ u_{2} \end{bmatrix}
    = \begin{bmatrix} \frac{F_{T}}{m} \\ \omega \end{bmatrix}
    \in {\rm I\!R}^{2}
\end{equation}

\begin{equation}
  \begin{bmatrix}
    \ddot{x} \\
    \ddot{z} \\
    \theta
  \end{bmatrix} =
  \begin{bmatrix}
    \cos(\theta) & -\sin(\theta) & 0 \\
    \sin(\theta) & \cos(\theta) & 0 \\
    0 & 0 & 1
  \end{bmatrix}
  \begin{bmatrix} 0 \\ u_{1} \\ u_{2} \end{bmatrix} +
  \begin{bmatrix} 0 \\ -g \\ 0 \end{bmatrix}
\end{equation}

\begin{equation}
  \dot{\textbf{x}} =
    \begin{bmatrix}
      \dot{x_{1}} \\ \dot{x_{2}} \\ \dot{x_{3}} \\ \dot{x_{4}} \\ \dot{x_{5}}
    \end{bmatrix} =
    f(\textbf{x}, \textbf{u}) =
    \begin{bmatrix}
      x_{2} \\
      u_{1} \sin(x_{5}) \\
      x_{4} \\
      u_{1} \cos(x_{5}) - g \\
      u_{2}
    \end{bmatrix}
\end{equation}

\begin{equation}
  \textbf{u}(t) \in \textbf{U} = \\
\end{equation}

% SYSTEM ARCHITECTURE
\section{System Architecture}


% EXPERIMENT RESULTS
\section{Experiment Results}



% CONCLUSIONS
\section{Conclusions and Future Work}


\bibliography{paper}


\end{document}
